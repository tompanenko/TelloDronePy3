\documentclass[12pt]{article}
\input{preamble/common}
\input{preamble/courseProblemSet}
\usepackage{listings}

\title{Tello Drones Workshop}
\author{Roberto Tron}
\date{\today}

\newcommand{\textsfbf}[1]{\textsf{\textbf{#1}}}
\newcommand{\info}[1]{\par\textcolor{gray}{\textsfbf{Info:} \emph{#1}}}
\newcommand{\trythis}[1]{\par\textcolor{DodgerBlue4}{\textsfbf{Try this:} \emph{#1}}}
\newcommand{\warning}[1]{\par\textcolor{red}{\textsfbf{Note:} \emph{#1}}}

\begin{document}
\maketitle

\section{First programming steps}

\subsection{Opening the editor}
Open the IDLE editor by double-clicking on the icon on the desktop. We will use this simple program to write and execute our code.
\info{Programs that allow you to write, organize, and run code, all without leaving that program, are called Integrated Development Environments (IDE).}

\subsection{Python}
Pythons are serpents, but Python (with the capital and no ``s'') is a programming language. Nowadays, Python is among the most popular programming languages; this is probably due to the low difficulty in getting started, and the fact that its functionality can be extended with a large number of \emph{modules}.

Python is an \emph{intepreted language}. In practice, this means that you can program in one of two ways:
\begin{enumerate}
\item Write commands at a \emph{prompt}. This is what you see after opening IDLE. After typing a command, it gets immediately executed (giving an error if it is not correct). Afterward, it the prompt waits for another command, and the cycle repeats until you exit.
\item Write commands in a file, which is called a \emph{script}. You can then execute all the commands one after the other by \emph{running} the script. This is equivalent to typing each command at the prompt, although the execution of the script stops if it encounters an error.
\end{enumerate}

\subsection{Hello world}
At the IDLE prompt, type
\begin{lstlisting}
print('Hello world!')
\end{lstlisting}

The \var{''} characters delimit a \emph{string}, which is simply a collection of characters. \var{print} is a \emph{function}; functions take in \emph{arguments} (in this case, a string) to perform some action (in this case, showing it on the screen). Sometimes, it also \emph{returns} a result (more on this later).

\trythis{Select the menu \emph{File}, \emph{New} to open a new file. Write a sequence of print statements. Save and run the script.}

\subsection{Variables}
\emph{Variables} are like named labels that you put on specific \emph{objects} (items) in memory. For instance, the code
\begin{lstlisting}
  a='Hello world!'
\end{lstlisting}
gives the label \var{a} to the string \var{'Hello world'} in memory. \info{Sometimes, you might also hear that the variable \var{a} \emph{contains} the string.}

You can then use \var{a} instead of using the object directly. For instance,
\begin{lstlisting}
  print(a)
\end{lstlisting}
produces the same result as in the previous section.

\subsection{Modules and extending functionality with imports}
You can extend functionality by \emph{importing modules}. For instance, the Python language by itself cannot make the drone fly. However, you can import functionality as in this example:
\begin{lstlisting}
  from djitellopy import Tello
\end{lstlisting}
In this case, we import \var{Tello} (which is a \emph{class}, more on this later) from the module \var{djitellopy}.

You can then used the functions or classes that you imported:
\begin{lstlisting}
  tello_object=Tello()
\end{lstlisting}

\info{Names in Python are case-sensitive, so \var{tello} and \var{Tello} are recognized as different things.}

\subsection{Objects}
Objects are essentially pieces of memory with a prescribed organization. Objects can contain variables (which are labels for other objects) or functions; \emph{classes} define what variables and functions go inside an object. For instance, in the previous example
\begin{lstlisting}
  tello_object=Tello()
\end{lstlisting}
you created an object \var{tello_object} of class \var{Tello}.

You can access variables or functions inside an object as in the example below.

\trythis{Before running the example, you will need to turn on the drone and connect to its WiFi network. It is best to use run the example as a script.}

\begin{lstlisting}
  from djitellopy import Tello
  tello_object=Tello()
  
  tello_object.connect()
  tello_object.end()
\end{lstlisting}

As shown in the example, you need to use the name of the object \var{tello_object} followed by dot \var{.}, followed by the function or variable name (in this case, \var{connect()} first, and then \var{end()}).

\info{To be precise, \var{tello_object} is a variable pointing to the object, but in general this subtle distinction is omitted.}

\info{When using the drone, you need to always call the \var{end()} function, otherwise you might have trouble to connect to the drone again (this can be fixed, but it requires a little bit of time).}


\section{The state of the drone}

You can see the state of the drone (roll/pitch/yaw, accelerometer readings, and other information) by using the function \var{print_state()} in the drone.

\begin{lstlisting}
  from djitellopy import Tello
  tello=Tello()
  
  tello.connect()
  tello.print_state()
  tello.end()
\end{lstlisting}

\subsection{More programming: repeating things (loops)}
Sometimes, you would like to repeat a command more than once. For instance, you would like to show the state multiple times for each execution. This can be done with a \emph{for loop}. The easiest way to write for loops in Python is as in this example:
\begin{lstlisting}
  for count in range(0,4):
      #commands in the loop to be repeated
      #all commands to be repeated should be indented with spaces in front
      #the number of spaces in front is usually 4
\end{lstlisting}

\info{Do not forget the semicolon in the first line, otherwise you will get an error.}

In this case, the commands inside the loop are repeated four times; each time, the variable \var{count} will have one of the values \var{0}, \var{1}, \var{2}, \var{3}. How many times the loop gets repeated and what values the variable will cycle through depend on the arguments to the function \var{range} (which are \var{0} and \var{4})

\trythis{What do you think the following commands will do?}
\begin{lstlisting}
  for count in range(0,5):
      print(count)
\end{lstlisting}

\subsection{Waiting a given amount of time}
By default, for loops get executed as fast as Python can run the commands. However, you can insert pauses by using the function \var{sleep(sec)} from the \var{time} module; \var{sec} specifies for how many seconds the execution should wait.

\trythis{This example is similar to the previous, but with two-seconds pauses after each repetition of the loop.}
\begin{lstlisting}
  from time import sleep
  for count in range(0,5):
      print(count)
      sleep(2)
\end{lstlisting}

\subsection{Reading the state of the drone multiple times}
\trythis{We can put together all the material in this subsection}
\begin{lstlisting}
  from djitellopy import Tello
  from time import sleep
  
  tello=Tello()
  
  tello.connect()
  for count in range(0,5):
      tello.print_state()
      sleep(1.5)
  tello.end()
\end{lstlisting}

\section{Flying the drone}

\subsection{Takeoff and landing}
You can have the drone takeoff and land by calling the functions \var{takeoff()} and \var{land()} in the \var{tello} object.
\begin{lstlisting}
  from djitellopy import Tello
  from time import sleep
  
  tello=Tello()
  
  tello.connect()
  tello.takeoff()
  sleep(1.5)
  tello.land()
  tello.end()
\end{lstlisting}

\subsection{Moving (translation)}
You can move in any direction (up,down,left,right,forward,back) by using the corresponding functions in the \var{tello} object.
\info{The argument to the function must be a distance in \unit{cm} higher than 20 and no lower than 100}
\begin{lstlisting}
  from djitellopy import Tello

  tello = Tello()

  tello.connect()
  tello.takeoff()
  tello.move_up(30)
  tello.move_down(30)
  tello.move_forward(30)
  tello.move_back(30)
  tello.move_left(30)
  tello.move_right(30)
  tello.land()
\end{lstlisting}

\subsection{Rotating}
You can change the yaw of the drone with the functions \var{rotate_clockwise(deg)} and \var{rotate_counter_clockwise(deg)}; the argument \var{deg} must be a number of degrees between 1 and 360.

\begin{lstlisting}
  from djitellopy import Tello

  tello = Tello()

  tello.connect()
  tello.takeoff()
  tello.rotate_clockwise(180)
  tello.rotate_counter_clockwise(180)
  tello.land()
\end{lstlisting}

\subsection{Putting things together}
\info{For the questions below, it is easier to first write the scripts by using \var{print} to show the commands on screen instead actually executing them (e.g., \var{print('tello.move_up(50)')}). After you are satisfied with the sequencing, and when there are no errors, then you can substitute the \var{print}}.

\trythis{Can you write a script such that the drone traces an imaginary square?}
\trythis{Can you write a script such that the drone traces an imaginary square four times, with pauses of 2 seconds in between?}
\trythis{Can you write a script that traces the edges of the side faces of an imaginary cube?}


\section{Working with images}

\subsection{More on importing modules}
You can also import entire modules, and then decide what functions or classes to use from them later. For instance, the module \var{cv2} contains many functions to handle images. 
For instance:
\begin{lstlisting}
  import cv2
  img=cv2.imread('BU_logo.png')
\end{lstlisting}
As shown in this example, functions inside the module (e.g., \var{imread}) can then be called by using the name of the module \var{cv2} and a dot \var{.} before their name. 

\subsection{Loading and showing images}
The following expanded example loads an image with \var{imread}, shows it on screen with \var{imshow}, waits for a key to be pressed (\textbf{Note: you need to press the key while the image window, not the prompt, is in focus}), then closes all the image windows.
\info{Calling \var{imshow} does not show the image window immediately. It is necessary to use \var{waitKey} to make it appear.}
\info{If you do not call \var{destroyAllWindows}, the window will remain in a ``Not responding'' state, so please always remember to call it before the end of the script.}
\begin{lstlisting}
  import cv2
  img=cv2.imread('BU_logo.png')
  cv2.imshow('Image title',img)
  cv2.waitKey(0)
  cv2.destroyAllWindows()
\end{lstlisting}

\subsection{Taking images from the drone and saving them}
To be able to get images from the camera onboard the drone, you need to use the function \var{stream_on()} on the \var{tello} object; this function should be called only once right after the call to the function \var{connect()}.
The actual images can then be obtained by calling the function \var{tello.get_frame()}.
You can save the images on disk with the function \var{imwrite('filename.png',img)} from the \var{cv2} module (change \var{filename} to the name of the file that you prefer, but do not forget the extension \var{.png}).

The following example summarizes all these commands:
\begin{lstlisting}
from djitellopy import Tello
import cv2
tello.connect()
tello.stream_on()
img = tello.get_frame()
cv2.imwrite('filename.png',img)
tello.end()
\end{lstlisting}
\warning{When running this script, please wait for it to end completely (it might take a few seconds). You should see \var{update_frame: terminating} in the prompt window. If you close the prompt before this, you might have trouble getting images.}

\info{The image should get saved in the \emph{Documents/djitello} directory.}

\trythis{Take a selfie with the drone.}
\trythis{Can you write a script to make the drone fly and then take a sequence of pictures in four or more directions?}

\subsection{Debugging image acquisition problems}
If there are problems in the system (e.g., a previous script crashed before the call to \var{tello.end()}), the variable \var{img} might be empty. The following example adds a \var{if} command to check for this, and print a warning if it happens.

\begin{lstlisting}
from djitellopy import Tello
import cv2
tello.connect()
tello.stream_on()
img = tello.get_frame()
if not img:
    print('No frame')
else:
    cv2.imwrite('filename.png',img)
tello.end()
\end{lstlisting}


\end{document}
